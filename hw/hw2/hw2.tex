\documentclass[12pt,letterpaper]{article}

\newenvironment{proof}{\noindent{\bf Proof:}}{\qed\bigskip}

\newtheorem{theorem}{Theorem}
\newtheorem{corollary}{Corollary}
\newtheorem{lemma}{Lemma} 
\newtheorem{claim}{Claim}
\newtheorem{fact}{Fact}
\newtheorem{definition}{Definition}
\newtheorem{assumption}{Assumption}
\newtheorem{observation}{Observation}
\newtheorem{example}{Example}
\newcommand{\qed}{\rule{7pt}{7pt}}

\newcommand{\assignment}[4]{
\thispagestyle{plain} 
\newpage
\setcounter{page}{1}
\noindent
\begin{center}
\framebox{ \vbox{ \hbox to 6.28in
{\bf CS421 \hfill #1}
\vspace{4mm}
\hbox to 6.28in
{\hspace{2.5in}\large\mbox{Problem Set #2}}
\vspace{4mm}
\hbox to 6.28in
{{\it Handed Out: #3 \hfill Due: #4}}
}}
\end{center}
}

\newcommand{\solution}[4]{
\thispagestyle{plain} 
\newpage
\setcounter{page}{1}
\noindent
\begin{center}
\framebox{ \vbox{ \hbox to 6.28in
{\bf CS421 : hw#3 \hfill #4}
\vspace{4mm}
{#1 \hfill {\it Handed In: #2}}
}}
\end{center}
\markright{#1}
}

\newenvironment{algorithm}
{\begin{center}
\begin{tabular}{|l|}
\hline
\begin{minipage}{1in}
\begin{tabbing}
\quad\=\qquad\=\qquad\=\qquad\=\qquad\=\qquad\=\qquad\=\kill}
{\end{tabbing}
\end{minipage} \\
\hline
\end{tabular}
\end{center}}

\def\Comment#1{\textsf{\textsl{$\langle\!\langle$#1\/$\rangle\!\rangle$}}}


\usepackage{algorithm}
\usepackage{listings}
%\usepackage{algpseudocode}
\usepackage{graphicx,amssymb,amsmath}
\usepackage{mathtools}
\usepackage{epstopdf}
\usepackage{color}
\usepackage[dvipsnames]{xcolor}
\sloppy

\oddsidemargin 0in
\evensidemargin 0in
\textwidth 6.5in
\topmargin -0.5in
\textheight 9.0in

\begin{document}

\solution{Dan McQuillan}{\today}{2}{Summer 2015}

\pagestyle{myheadings}  % Leave this command alone
	 			
		
\begin{enumerate}	
\item{\bf{Problem 1}} \\ \\
let p1 = (1., 3.);; \\
let p2 = (2., 5.);;

\begin{flalign*}
\rho_1 = & \{  & \\
	& \hspace*{8mm} p2 \rightarrow (2., 5.),  \\ 
	& \hspace*{8mm} p1 \rightarrow (1., 3.) \\
 & \}  &
 \end{flalign*}

let slope (x,y) = y /. x;;

\begin{flalign*}
\rho_2 = & \{  & \\
	& \hspace*{8mm} slope \rightarrow <(x,y) \rightarrow y /. x, \rho_1>, \\ 
	& \hspace*{8mm} p2 \rightarrow (2., 5.),  \\ 
	& \hspace*{8mm} p1 \rightarrow (1., 3.) \\
 & \}  &
 \end{flalign*}

let sub (x1,y1) (x2,y2) = (x2 - x1, y2 - y1);;

\begin{flalign*}
\rho_3 = & \{  & \\
	& \hspace*{8mm} sub \rightarrow <(x1,y1) \rightarrow  \text{fun }  (x2, y2) \rightarrow (x2 - x1, y2 - y1), \rho_2>, \\ 
	& \hspace*{8mm} slope \rightarrow <(x,y) \rightarrow y /. x, \rho_1>, \\ 
	& \hspace*{8mm} p2 \rightarrow (2., 5.),  \\ 
	& \hspace*{8mm} p1 \rightarrow (1., 3.) \\
 & \}  &
\end{flalign*}

let slope p1 p2 = slope (sub p1 p2);;

\begin{flalign*}
\rho_4 = & \{  & \\
	& \hspace*{8mm} slope \rightarrow <p1 \rightarrow \text{fun } p2 \rightarrow \text{slope( sub p1 p2 )}, \rho_3>, \\ 
	& \hspace*{8mm} sub \rightarrow <(x1,y1) \rightarrow  \text{fun }  (x2, y2) \rightarrow (x2 - x1, y2 - y1), \rho_2>, \\ 
	& \hspace*{8mm} p2 \rightarrow (2., 5.),  \\ 
	& \hspace*{8mm} p1 \rightarrow (1., 3.) \\
 & \}  &
\end{flalign*}

let slope\_p2 = slope p2;;


Evaluation:

\begin{flalign*}
	\text{Eval}(\text{slope p2}, \rho_4) &= \text{Eval}(\text{App}(<p1 \rightarrow \text{fun } p2 \rightarrow \text{slope( sub p1 p2 )}, \rho_3>, (2., 5.)), \rho_4) \\
	&= \text{Eval}( < \text{fun } p2 \rightarrow \text{slope( sub p1 p2 )}, \left\{ p1 \rightarrow (2., 5.) \right\} + \rho_3>, \rho_4) \\
	&= <p2 \rightarrow \text{slope(sub p1 p2)}, \left\{ p1 \rightarrow (2., 5.) \right\} + \rho_3>
\end{flalign*}

\begin{flalign*}
\rho_5 = & \{  & \\
	& \hspace*{8mm} slope\_p2 \rightarrow <p2 \rightarrow \text{slope(sub p1 p2)}, \left\{ p1 \rightarrow (2., 5.) \right\} + \rho_3> \\
	& \hspace*{8mm} slope \rightarrow <p1 \rightarrow \text{fun } p2 \rightarrow \text{slope( sub p1 p2 )}, \rho_3>, \\ 
	& \hspace*{8mm} sub \rightarrow <(x1,y1) \rightarrow  \text{fun }  (x2, y2) \rightarrow (x2 - x1, y2 - y1), \rho_2>, \\ 
	& \hspace*{8mm} p2 \rightarrow (2., 5.),  \\ 
	& \hspace*{8mm} p1 \rightarrow (1., 3.) \\
 & \}  &
\end{flalign*}

let p2 = (3., 9.);;

\begin{flalign*}
\rho_6 = & \{  & \\
	& \hspace*{8mm} p2 \rightarrow (3., 9.),  \\ 
	& \hspace*{8mm} slope\_p2 \rightarrow <p2 \rightarrow \text{slope(sub p1 p2)}, \left\{ p1 \rightarrow (2., 5.) \right\} + \rho_3> \\
	& \hspace*{8mm} slope \rightarrow <p1 \rightarrow \text{fun } p2 \rightarrow \text{slope( sub p1 p2 )}, \rho_3>, \\ 
	& \hspace*{8mm} sub \rightarrow <(x1,y1) \rightarrow  \text{fun }  (x2, y2) \rightarrow (x2 - x1, y2 - y1), \rho_2>, \\ 
	& \hspace*{8mm} p1 \rightarrow (1., 3.) \\
 & \}  &
\end{flalign*}

slope\_p2 p1;;

\begin{flalign*}
	\text{slope\_p2 p1} &= \text{Eval}(\text{slope\_p2 p1}, \rho_6) \\
	&= \text{Eval}(\text{App}(<p2 \rightarrow \text{slope(sub p1 p2)}, \left\{ p1 \rightarrow (2., 5.) \right\} + \rho_3>, (1., 3.)), \rho_6) \\
	&= \text{Eval}(\text{slope(sub p1 p2)}, \left\{ p2 \rightarrow (1., 3.), p1 \rightarrow (2., 5.) \right\} + \rho_3) \\
	&= \text{Eval}(\text{App}( <(x,y) \rightarrow y /. x, \rho_1>, \\
	& \hspace*{15mm}  \text{Eval}(\text{sub p1 p2}, \left\{ p2 \rightarrow (1., 3.), p1 \rightarrow (2., 5.) \right\} + \rho_3 ))) \\
	&= \text{Eval}(\text{App}( <(x,y) \rightarrow y /. x, \rho_1>, \\
	& \hspace*{15mm} \text{Eval}( \text{App}( \text{App}(  \\
	& \hspace*{30mm} <(x1,y1) \rightarrow  \text{fun }  (x2, y2) \rightarrow (x2 - x1, y2 - y1), \rho_2>, \\
	& \hspace*{15mm}  (2., 5.) ), (1., 3.) ) ))) \\
	&= \text{Eval}(\text{App}( <(x,y) \rightarrow y /. x, \rho_1>, \\
	& \hspace*{15mm} \text{Eval}( (x2 - x1, y2 - y1) , \left\{ x2 \rightarrow 1, y2 \rightarrow 3, x1 \rightarrow 2, y1 \rightarrow 5 \right\} + \rho_2))) \\
	&= \text{Eval}(\text{App}( <(x,y) \rightarrow y /. x, \rho_1>, (-1, -2)))\\
	&= \text{Eval}(y /. x, \left\{ x \rightarrow -1, y \rightarrow -2 \right\} + \rho_1 )\\
	&= 2
\end{flalign*}

slope p1 p2;;

\begin{flalign*}
	\text{slope p1 p2} &= \text{Eval}(\text{slope p1 p2}, \rho_6) \\
	&= \text{Eval}(\text{App}(\text{App}(<p1 \rightarrow \text{fun } p2 \rightarrow \text{slope(sub p1 p2)}, \rho_3>, (1., 3.)), (3., 9.))) \\
	&= \text{Eval}(\text{slope(sub p1 p2)}, \left\{ p2 \rightarrow (3., 9.), p1 \rightarrow (1., 3.) \right\} + \rho_3) \\
	&= \text{Eval}(\text{App}( <(x,y) \rightarrow y /. x, \rho_1>, \\
	& \hspace*{15mm}  \text{Eval}(\text{sub p1 p2}, \left\{ p2 \rightarrow (3., 9.), p1 \rightarrow (1., 3.) \right\} + \rho_3 ))) \\
	&= \text{Eval}(\text{App}( <(x,y) \rightarrow y /. x, \rho_1>, \\
	& \hspace*{15mm} \text{Eval}( \text{App}( \text{App}(  \\
	& \hspace*{30mm} <(x1,y1) \rightarrow  \text{fun }  (x2, y2) \rightarrow (x2 - x1, y2 - y1), \rho_2>, \\
	& \hspace*{15mm}  (1., 3.) ), (3., 9.) ) ))) \\
	&= \text{Eval}(\text{App}( <(x,y) \rightarrow y /. x, \rho_1>, \\
	& \hspace*{15mm} \text{Eval}( (x2 - x1, y2 - y1) , \left\{ x2 \rightarrow 3, y2 \rightarrow 9, x1 \rightarrow 1, y1 \rightarrow 3 \right\} + \rho_2))) \\
	&= \text{Eval}(\text{App}( <(x,y) \rightarrow y /. x, \rho_1>, (2, 6)))\\
	&= \text{Eval}(y /. x, \left\{ x \rightarrow 2, y \rightarrow 6 \right\} + \rho_1 )\\
	&= 3
\end{flalign*}

\pagebreak

\item{\bf{Problem 2}} \\ \\

\begin{lstlisting}
let f g x =
  (let r =
    if ((print_string "a"; x > 5) && (g(); x > 10))
    then
      (print_string "b"; x - 7)
    else
      let z = (print_string "c"; 15) in (print_string "d"; z)
    in (g(); r));;
let u = (f (fun () -> print_string "e\n") 
           (f (fun () -> print_string "f\n") 3));;
\end{lstlisting}

The first command creates a curried function, f.  This function has the following type information:
val f : (unit \(\rightarrow\) 'a) \(\rightarrow\) int \(\rightarrow\) int = \(<\)fun\(>\)

The instantiation of u prints the following:
acdf\\
ae\\
be\\
val u : int = 8

u is initialized by f which is done by first invoking a call to the function, f.  f has 2 parameters that need to be passed to it, g and x.

g is a function with no parameters and simply prints out "e" and then a trailing new line and will finally return unit.
x will represent a second call to the function and will ultimately return a value, to be utilized by the function that is initializing u as mentioned above.  The first parameter prints out "f" and then a trailing new line and will then return unit. 

The logic within the function, f, will first print out the character "a".  Then if x is greater then 5 it will invoke the function g  and then if x is also greater than 10 it will print out the character "b" as well as return the value x reduced by 7 which will then be the value of r.  If the number is less than 10, then it will create a new variable z and in it's initialization it will print out "c" then that block will return 15 and the value for z will be 15 in the block: (print\_string "d"; z).  When that block is evaluated it will print the character "d" and then return the value z  which will be the value of r.

Finally, once we have the value of r from one of the two paths it will invoke the function g and then return the value of r.

Therefore the order of execution is the following for our script:
\begin{enumerate}
	\item Evaluate \textbf{(f (fun () \(\rightarrow\) print\_string "f\textbackslash n") 3)}
		\begin{enumerate}
			\item Call function f with params (fun () \(\rightarrow\) print\_string "f\textbackslash n"), 3 
			\begin{enumerate}
				\item print out string "a" 
				
				\item 3 < 5 therefore we go to else block 
				
				\item print out string "c" 
				
				\item initialize z to 15 
				
				\item print out string "d" 
				
				\item initialize r to z which is 15 
				
				\item call function g which will print out "f" and a new line 
				
				\item return the value of r which is 15 from the function f 
			\end{enumerate}
			After the above flows we will have on the console: \\
			 "acdf\textbackslash n"
			\item Get the return value of the function call to f which will be 15 
		\end{enumerate}
	\item Evaluate \textbf{f (fun () \(\rightarrow\) print\_string "e\textbackslash n") 15 }
		\begin{enumerate}
			\item Call function f with params (fun() \(\rightarrow\) print\_string "e\textbackslash n") 15 
			\begin{enumerate}
				\item print out string "a" 
				\item 15 > 5 therefore we continue by evaluating the expression after the \&\& 
				\item invoke g which will print out the string "e\textbackslash n" 
				\item 15 > 10 therefore we continue into the primary block of the ternary expression 
				\item print out string "b" and return 15 - 7 from the ternary expression 
				\item initialize r to 8 
				\item invoke g which will print out the string "e\textbackslash n" 
				\item return the value of r which will be 8 from the function f 
			\end{enumerate}
			After the above steps we will have on the console: \\
			acdf \\
			ae \\
			be \\
		\end{enumerate}
	\item{ Evaluate let u = 8;; }  This will set up a variable u in our environment with a value of 8.			
\end{enumerate}

Our final content in our console will be what was described above: \\
acdf\\
ae\\
be\\
val u : int = 8
\end{enumerate}


\end{document}

