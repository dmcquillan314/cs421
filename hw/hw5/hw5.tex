\documentclass[12pt,letterpaper]{article}

\newenvironment{proof}{\noindent{\bf Proof:}}{\qed\bigskip}

\newtheorem{theorem}{Theorem}
\newtheorem{corollary}{Corollary}
\newtheorem{lemma}{Lemma} 
\newtheorem{claim}{Claim}
\newtheorem{fact}{Fact}
\newtheorem{definition}{Definition}
\newtheorem{assumption}{Assumption}
\newtheorem{observation}{Observation}
\newtheorem{example}{Example}
\newcommand{\qed}{\rule{7pt}{7pt}}

\newcommand{\assignment}[4]{
\thispagestyle{plain} 
\newpage
\setcounter{page}{1}
\noindent
\begin{center}
\framebox{ \vbox{ \hbox to 6.28in
{\bf CS421 \hfill #1}
\vspace{4mm}
\hbox to 6.28in
{\hspace{2.5in}\large\mbox{Problem Set #2}}
\vspace{4mm}
\hbox to 6.28in
{{\it Handed Out: #3 \hfill Due: #4}}
}}
\end{center}
}

\newcommand{\solution}[4]{
\thispagestyle{plain} 
\newpage
\setcounter{page}{1}
\noindent
\begin{center}
\framebox{ \vbox{ \hbox to 6.28in
{\bf CS421 : hw#3 \hfill #4}
\vspace{4mm}
{#1 \hfill {\it Handed In: #2}}
}}
\end{center}
\markright{#1}
}

\newenvironment{algorithm}
{\begin{center}
\begin{tabular}{|l|}
\hline
\begin{minipage}{1in}
\begin{tabbing}
\quad\=\qquad\=\qquad\=\qquad\=\qquad\=\qquad\=\qquad\=\kill}
{\end{tabbing}
\end{minipage} \\
\hline
\end{tabular}
\end{center}}

\def\Comment#1{\textsf{\textsl{$\langle\!\langle$#1\/$\rangle\!\rangle$}}}


\usepackage{algorithm}
\usepackage{listings}
%\usepackage{algpseudocode}
\usepackage{graphicx,amssymb,amsmath}
\usepackage{epstopdf}
\sloppy

\oddsidemargin 0in
\evensidemargin 0in
\textwidth 6.5in
\topmargin -0.5in
\textheight 9.0in

\begin{document}

\solution{Dan McQuillan}{\today}{5}{Summer 2015}

\pagestyle{myheadings}  % Leave this command alone
	 	
\section{Problems}

\subsection{Problem 1}
Give a most general unifier for the following set of equations (unification problem). The uppercase letters A, B, C, and D denote variables of unification. The lowercase letters f, g, and h are term constructors of arity 2, 3, and 1 respectively (i.e. take two, three or one argument(s), respectively). Show all your work by listing the operations performed in each step of the unification and the result of that step.

\[
	\text{Unify}\{(f(A, g(B, C, h(D))) = f(g(C, B, C), A))\}
\]

\subsubsection{Steps to solution}

\begin{enumerate}
	\item[1.]
	
	Pick a pair: \( (f(A, g(B,C,h(D) ) ) = f(g(C,B,C), A ) )\)
	
	Decompose: \\
	\hspace*{8mm}becomes: \( \{ A = g(C,B,C); g(B,C,h(D)) = A \} \) \\
	\hspace*{4mm}\( = \text{Unify} \{ A = g(C,B,C); g(B,C,h(D)) = A \} \)		
	
	\item[2.]
	
	Pick a pair: \( (g(B,C,h(D)) = A) \)
	
	Orient: \\
	\hspace*{4mm}\( = \text{Unify} \{ A = g(C,B,C); A = g(B,C,h(D)) \} \)	
		
	\item[3.]
	
	Pick a pair: \( (A = g(C,B,C) \)
	
	Eliminate: A with substitution \( \{ A \rightarrow g(C,B,C) \} \) \\
	\hspace*{4mm}\( = \text{Unify} \{ g(C,B,C) = g(B,C,h(D)) \} o \{ A \rightarrow g(C,B,C) \} \)		
		
	\item[4.]
	
	Pick a pair: \( (g(C,B,C) = g(B,C,h(D)) ) \)
	
	Decompose: \\
	\hspace*{8mm}becomes: \( \{ C = B; B = C; C = h(D)  \} \) \\
	\hspace*{4mm}\( = \text{Unify} \{ C = B; B = C; C = h(D) \} o \{ A \rightarrow g(C,B,C) \} \)		
		
	\item[5.]
	
	Pick a pair: \( ( B = C ) \)
	
	Orient: \\
	\hspace*{4mm}\( = \text{Unify} \{ C = B; C = B; C = h(D) \} o \{ A \rightarrow g(C,B,C) \} \)		
		
	\item[6.]
	
	Pick a pair: \( ( C = B ) \)
	
	\begin{enumerate}
	\item[6.1]
	
	Eliminate: C with substitution \( \{ C \rightarrow B \} \) \\
	\hspace*{4mm}\( = \text{Unify} \{ B = B; B = h(D) \} o \{ C \rightarrow B \} o \{ A \rightarrow g(C,B,C) \} \)
	
	\item[6.2]
	
	Compose Substitutions: \\
	\hspace*{4mm}\( = \text{Unify} \{ B = B; B = h(D) \} o \{ C \rightarrow B; A \rightarrow g(B, B, B) \} \)
	\end{enumerate}
	
	\item[7.]
	
	Pick a pair: \( (B = B) \)
	
	Delete \\
	\hspace*{4mm}\( = \text{Unify} \{ B = h(D) \} o \{ C \rightarrow B; A \rightarrow g(B, B, B) \} \)

	\item[8.]
	
	Pick a pair:  \( (B = h(D)) \)
	
	\begin{enumerate}
	
    	\item[8.1.]
    	
    	Eliminate: B with substitution \( \{ B \rightarrow h(D) \} \) \\
    	\hspace*{4mm}\( = \text{Unify} \{ \} o \{ B \rightarrow h(D) \} o \{ C \rightarrow B; A \rightarrow g(B, B, B) \} \)
    	
    	\item[8.2.]
    	
    	Compose Substitutions: \\
    	\hspace*{4mm}\( = \text{Unify} \{ \} o \{ B \rightarrow h(D); C \rightarrow h(D); A \rightarrow g(h(D), h(D), h(D) ) \} \)
	\end{enumerate}
	
	\item[9.]
	
	Unify is evaluating identity substitution \\
    	\hspace*{4mm}\( = \{ B \rightarrow h(D); C \rightarrow h(D); A \rightarrow g(h(D), h(D), h(D) ) \} \)
\end{enumerate}

\subsubsection{Solution}

\[
	\text{Unify}\{(f(A, g(B, C, h(D))) = f(g(C, B, C), A))\} = \{ B \rightarrow h(D); C \rightarrow h(D); A \rightarrow g(h(D), h(D), h(D) ) \}
\]

\subsubsection{Check Solution}

\begin{align} % requires amsmath; align* for no eq. number
   		     f(A, g(B, C, h(D))) {}&= f(g(C,B,C), A) \\
   \rightarrow  f(g(h(D),h(D), h(D)), g(h(D), h(D), h(D) ) ) {}&= f( g(h(D),h(D), h(D)), g(h(D), h(D), h(D) ) )
\end{align}

By applying a simultaneous substitution we verify that the equality holds when the substitution is applied to the constraint in the original unification problem.

\end{document}

