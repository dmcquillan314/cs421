\documentclass[12pt,letterpaper]{article}

\newenvironment{proof}{\noindent{\bf Proof:}}{\qed\bigskip}

\newtheorem{theorem}{Theorem}
\newtheorem{corollary}{Corollary}
\newtheorem{lemma}{Lemma} 
\newtheorem{claim}{Claim}
\newtheorem{fact}{Fact}
\newtheorem{definition}{Definition}
\newtheorem{assumption}{Assumption}
\newtheorem{observation}{Observation}
\newtheorem{example}{Example}
\newcommand{\qed}{\rule{7pt}{7pt}}

\newcommand{\assignment}[4]{
\thispagestyle{plain} 
\newpage
\setcounter{page}{1}
\noindent
\begin{center}
\framebox{ \vbox{ \hbox to 6.28in
{\bf CS421 \hfill #1}
\vspace{4mm}
\hbox to 6.28in
{\hspace{2.5in}\large\mbox{Problem Set #2}}
\vspace{4mm}
\hbox to 6.28in
{{\it Handed Out: #3 \hfill Due: #4}}
}}
\end{center}
}

\newcommand{\solution}[4]{
\thispagestyle{plain} 
\newpage
\setcounter{page}{1}
\noindent
\begin{center}
\framebox{ \vbox{ \hbox to 6.28in
{\bf CS421 : hw#3 \hfill #4}
\vspace{4mm}
{#1 \hfill {\it Handed In: #2}}
}}
\end{center}
\markright{#1}
}

\newenvironment{algorithm}
{\begin{center}
\begin{tabular}{|l|}
\hline
\begin{minipage}{1in}
\begin{tabbing}
\quad\=\qquad\=\qquad\=\qquad\=\qquad\=\qquad\=\qquad\=\kill}
{\end{tabbing}
\end{minipage} \\
\hline
\end{tabular}
\end{center}}

\def\Comment#1{\textsf{\textsl{$\langle\!\langle$#1\/$\rangle\!\rangle$}}}


\usepackage{algorithm}
\usepackage{listings}
%\usepackage{algpseudocode}
\usepackage{graphicx,amssymb,amsmath}
\usepackage{epstopdf}
\sloppy

\oddsidemargin -0.5in
\evensidemargin -0.5in
\textwidth 7.5in
\topmargin -0.5in
\textheight 9.0in

\begin{document}

\solution{Dan McQuillan}{\today}{6}{Summer 2015}

\pagestyle{myheadings}  % Leave this command alone
	 	
\section{Theoretical Questions}

\subsection{Problem 1}

In both cases: \\
let \( \Sigma = \{ a, b, c \} \)

\subsubsection{The language where a occurs in every third position}

If we're assuming that the lack of any string also matches: \\
( (a \(\vee\) b \(\vee\) c) (a \(\vee\) b \(\vee\) c) a )* \\ \\ If we're assuming that the presence of a string is required to match: \\
( (a \(\vee\) b \(\vee\) c) (a \(\vee\) b \(\vee\) c) a )+ \\

\subsubsection{The language where each string contains exactly 3 c's}
(a \(\vee\) b)* c (a \(\vee\) b)* c (a \(\vee\) b)* c (a \(\vee\) b)*

\subsection{Problem 2}

\begin{enumerate}
	\item\(L_1 = \{ w | w \text{ starts with a symbol 0 and contains the symbol 1 at least once} \} \text{ where } \Sigma = \{ 0, 1 \} \)
	% regular
	
	\begin{enumerate}
		\item{Regular expression}
				
		0 ( 0 \(\vee\) 1 )* 1 ( 0 \(\vee\) 1 )*		
				
		\item{Regular grammar}
		
		\( <Language\_1> :: = 0 <ZeroOrOne> \) \\
		\( <ZeroOrOne>   :: = 0 <ZeroOrOne> \) \\
		\( <ZeroOrOne>   :: = 1 <ZeroOrOneOrEmpty> \) \\
		\( <ZeroOrOneOrEmpty> :: = 0 <ZeroOrOneOrEmpty> \) \\
		\( <ZeroOrOneOrEmpty> :: = 1 <ZeroOrOneOrEmpty> \) \\
		\( <ZeroOrOneOrEmpty> :: = \epsilon \)
	\end{enumerate}
	
	\item\(L_2 = \{ w | w \text{ contains an equal number of 0s and 1s} \} \text{ where } \Sigma = \{ 0, 1 \} \)
	% non-regular
	
	\begin{proof}
	By contradiction; assume the language \(L_2\) is regular. 
	
	Let n be the length guaranteed by the pumping lemma. Suppose we have a string \( w = 0^n 1^n \).  Then \( | w | = 2 n \geq n \) and \( w \in L_2 \).  
	
	Therefore, there exists strings x, y, and z such that w = xyz, \( | xy | \leq n, y \neq \epsilon \) and for any number i, \( xy^iz \in L_2 \). 
	
	Since, \( | x y | \leq n \) y must consist of only 0s.  However \(xy^2z = 0^{n + |y|}1^n \), and since \( |y| > 0 \), we have that \(xy^2z \notin L_2 \)
	
	Therefore, we have a contradiction and our language is not regular.
	\end{proof}
	
	\item\(L_3 = \{ w | \text{ the length of w is odd} \} \text{ where } \Sigma = \{ a, b \} \)
	% regular
	
	\begin{enumerate}
		\item{Regular expression}
		
		( a \(\vee\) b ) ( ( a \(\vee\)  b ) ( a \(\vee\) b ) )*
		\item{Regular grammar}
		
		\( <Language\_3> :: = a <AOrBOrEmpty> \)\\
		\( <Language\_3> :: = b <AOrBOrEmpty> \)\\
		\( <AOrBOrEmpty> :: = a <Language\_3> \)\\
		\( <AOrBOrEmpty> :: = b <Language\_3> \)\\
		\( <AOrBOrEmpty> :: = \epsilon \)
	\end{enumerate}
	
	\item\(L_4 = \{ w | w \text{ does not contain symbol a immediately followed by symbol b} \} \text{ where } \Sigma = \{ a, b, c \} \)
	
	\begin{enumerate}
		\item{Regular expression}
		
		\item{Regular grammar}
		
	\end{enumerate}
	
	\item\(L_5 = \{ w | \text{ the length of w is a perfect cube } \} \text{ where } \Sigma = \{ a, b, c \} \)
	% non-regular
	
	\begin{proof}
	By contradiction; assume the language \(L_5\) is regular.
	
	Let n be the length guaranteed by the pumping lemma.   
	
	Suppose \( w = a^{\frac{n^3}{3}} b^{\frac{n^3}{3}} c^{\frac{n^3}{3}} \).
	
	Since \( |w| = \frac{n^3}{3} + \frac{n^3}{3} + \frac{n^3}{3} = n^3 \), which is a perfect cube, \( w \in L_5 \).
	
	By the pumping lemma we know that we can split \( w = xyz \) s.t. the conditions of the pumping lemma hold.
	
	We know that:
	\[
		1 \leq | y | \leq | xy | \leq n
	\]
	
	Since for the pumping lemma to hold we also require that any amount of y terms in the middle to still let the expression hold. Therefore we know also that:
	\[
		xy^2z \in L_5
	\]
	
	Therefore we may assume that \( | xy^2z | \) is a perfect cube.  However we know that:
	
	\begin{align}
		n^3 {}& = | w | \\
		       {}& = | xyz | \\
		       {}& < | xy^2z | \\
		       {}& \leq n^3 + n \hspace*{8mm}\text{ since, } | y | \leq n \\
		       {}& < n^3 + 3n^2 + 3n + 1
	\end{align}
	
	In summary, we now know that:
	\[
		n^3 < | xy^2z | < n^3 + 3n^2 + 3n + 1
	\]
	
	That is \( | xy^2z | \) lies between two subsequent perfect cubes.  Therefore, it cannot be a perfect cube itself, and hence we have a contradiction to \( xyyz \in L_5 \).
	\end{proof}
\end{enumerate}

\end{document}

