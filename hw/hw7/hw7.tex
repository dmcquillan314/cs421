\input{../style/cs421_style.tex}
\usepackage{algorithm}
\usepackage{listings}
%\usepackage{algpseudocode}
\usepackage{booktabs}
\usepackage{mathtools}
\usepackage{graphicx,amssymb,amsmath}
\usepackage{epstopdf}
\usepackage{tikz}
\sloppy

\oddsidemargin -0.5in
\evensidemargin -0.5in
\textwidth 6.5in
\topmargin -0.5in
\textheight 9.0in

\DeclareMathSizes{10}{10}{10}{10}

\begin{document}

\solution{Dan McQuillan}{\today}{7}{Summer 2015}

\pagestyle{myheadings}  % Leave this command alone
	 	
\def \exptype { \( \langle exp \rangle \) }
\def \sumexp { \( \langle sum\_exp \rangle \) }
\def \atom { \( \langle atom \rangle \) }
\def \vartype { \( \langle var \rangle \) }
\def \while { \text{ while } }
\def \do { \text{ do } }
\def \od { \text{ od } }
\def \iftext { \text{ if } }
\def \fitext { \text{ fi } }
\def \thentext { \text{ then } }
\def \elsetext { \text{ else } }
\def \skip { \text{ skip } }
		
\begin{enumerate}

\item[1]

Consider the following grammar over the alphabet {if, then, else, +, x, y, z, (, )}:

\begin{enumerate}

\item[a]{Show that the above grammar is ambiguous by showing at least three distinct parse trees for the string "if x then x else x + y + z"}

\begin{tikzpicture}[level distance=1.2cm,
  level 1/.style={sibling distance=2.4cm},
  level 2/.style={sibling distance=1.2cm}]
\node { \exptype }
  child {node {if} }
  child {node {\exptype }
  	child { node {\vartype} 
		child { node { x } }
	}
  }
  child {node {then} }
  child {node {\exptype} 
  	child { node {\vartype} 
		child { node { x } }
	}
  }
  child {node {else} }
  child {node {\exptype} 
  	child { node { \exptype } 
  		child { node {\vartype} 
			child { node { x } }
		}
	}
	child { node { + } }
	child { node { \exptype } 
		child { node { \exptype }
			child { node { \vartype }
				child { node { y } }
			}
		}
		child { node { + } }
		child { node { \exptype }
			child { node { \vartype }
				child { node { z } }
			}
		}
	}
  };
\end{tikzpicture}

\begin{tikzpicture}[level distance=1.2cm,
  level 1/.style={sibling distance=2.4cm},
  level 2/.style={sibling distance=1.2cm}]
\node { \exptype }
  child {node {if} }
  child {node {\exptype }
  	child { node {\vartype} 
		child { node { x } }
	}
  }
  child {node {then} }
  child {node {\exptype} 
  	child { node {\vartype} 
		child { node { x } }
	}
  }
  child {node {else} }
  child {node {\exptype} 
	child { node { \exptype } 
		child { node { \exptype }
			child { node { \vartype }
				child { node { x } }
			}
		}
		child { node { + } }
		child { node { \exptype }
			child { node { \vartype }
				child { node { y } }
			}
		}
	}
	child { node { + } }
  	child { node { \exptype } 
  		child { node {\vartype} 
			child { node { z } }
		}
	}
  };
\end{tikzpicture}

\begin{tikzpicture}[level distance=1.2cm,
  level 1/.style={sibling distance=4.0cm},
  level 2/.style={sibling distance=2.0cm}]
\node { \exptype  }
	child { node { \exptype }
		child { node { if } }
            	child { node { \exptype } 
			child { node { \vartype }
				child { node { x } }
			}
		}
            	child { node { then } }
            	child { node { \exptype } 
			child { node { \vartype }
				child { node { x } }
			}
		}
            	child { node { else } }
            	child { node { \exptype } 
			child { node { \exptype }
				child { node { \vartype }
					child { node { x } }
				}
			}
			child { node { + } }
			child { node { \exptype }
				child { node { \vartype }
					child { node { y } }
				}
			}
		}
	}
	child { node { + } }
	child { node { \exptype } 
		child { node { \vartype }
			child { node { z } }
		}
	};
\end{tikzpicture}

\begin{tikzpicture}[level distance=1.2cm,
  level 1/.style={sibling distance=4.0cm},
  level 2/.style={sibling distance=2.0cm}]
\node { \exptype  }
	child { node { \exptype }
		child { node { if } }
            	child { node { \exptype } 
			child { node { \vartype }
				child { node { x } }
			}
		}
            	child { node { then } }
            	child { node { \exptype } 
			child { node { \vartype }
				child { node { x } }
			}
		}
            	child { node { else } }
            	child { node { \exptype } 
			child { node { \vartype }
				child { node { x } }
			}
		}
	}
	child { node { + } }
	child { node { \exptype } 
		child { node { \exptype }
			child { node { \vartype }
				child { node { y } }
			}
		}
		child { node { + } }
		child { node { \exptype }
			child { node { \vartype }
				child { node { z } }
			}
		}
	};
\end{tikzpicture}

\item[b]{Write a new grammar accepting the same language that is unabmiguous, and such that addition \(\langle exp \rangle + \langle exp \rangle \) has higher precedence than conditional.}

\begin{align}
	\langle exp \rangle ::= &{} \text{if } \langle sum\_exp \rangle \text{ then } \langle sum\_exp \rangle \text{ else } \langle sum\_exp \rangle  \\
				          &{} | \langle sum\_exp \rangle \\
	\langle sum\_exp \rangle ::= &{} \langle sum\_exp \rangle + \langle atom \rangle \\
					 &{} | \langle atom \rangle \\
	\langle atom \rangle ::= &{} \langle var \rangle \\
					 &{} | ( \langle exp \rangle ) \\
	\langle var \rangle ::= &{} x | y | z
\end{align}

\item[c]{ Give the parse tree for "f x then x else x + y + z" using the grammar you gave in the previous part of this problem. }

\begin{tikzpicture}[level distance=1.2cm,
  level 1/.style={sibling distance=2.4cm},
  level 2/.style={sibling distance=1.6cm}]
\node { \exptype  }
	child { node { if } }
	child { node { \sumexp }
		child { node { \atom }
			child { node { \vartype }
				child { node { x } }
			}
		}
	}
	child { node { then } }
	child { node { \sumexp }
		child { node { \atom }
			child { node { \vartype }
				child { node { x } }
			}
		}
	}
	child { node { else } }
	child { node { \sumexp }
		child { node { \sumexp }
			child { node { \sumexp }
				child { node { \atom }
					child { node { \vartype }
						child { node { x } }
					}
				}
			}
			child { node { + } }
			child { node { \atom }
				child { node { \vartype }
					child { node { y } }
				}
			}
		}
		child { node { + } }
		child { node { \atom } 
			child { node { \vartype } 
				child { node { z } }
			}
		}
	};
\end{tikzpicture}

\end{enumerate}

\item[2] Add a new increment operator ++I to the syntax of expression E and a new do-while operator do C while B od to the syntax of commands C

\( I \in \text{ Identifiers } \) \\
\( N \in \text{ Numerals } \) \\
\( B ::= \text{ true } | \text{ false } | B \text{ \& } B | B \text{ or } B | \text{ not } B | E < E | E = E \) \\
\( E ::= N | I | ++I | E + E | E * E | E - E | - E \) \\
\( C ::= \text{skip} | C;C | I ::= E | \text{ if } B \text{ then } C \text{ else } C \text{ fi } | \text{ while } B \text{ do } C \text{ od } | \text{ do } C \text{ while } B \text{ od } \) \\

a. Add the structural operational semantics (a.k.a. natural semantics) for these operators. Note that the operators work as follows. The semantics of the operator ++I is to add one to the current value of I, then store the new value into I. The execution of do C while B od starts with executing the command C in the body of the loop. The loop is repeated until the boolean expression B is evaluated to false. \\

Natural Semantics: \\

++I:
\[
	\frac
	{
		v = m(I) + 1
	}
	{
		(++I, m) \Downarrow (v, m[ I \rightarrow v ] )
	}
\]

do C while B od:
\[
	\frac
	{
		(C, m) \Downarrow m' \hspace*{4mm} (\text{ while } B \text{ do } C \text{ od }, m') \Downarrow m''
	}
	{
		(\text{do } C \text{ while } B \text{ od }, m) \Downarrow m''
	}
\]

b.  Add the transition semantics for these operators. They have the same meaning as part a. \\

Transition Semantics: \\

++I:
\[
	\frac
	{
		v = m(I) + 1
	}
	{
		(++I, m) \rightarrow (v, m[ I \rightarrow v ] )
	}
\]

do C while B od:
\[
	(\text{do } C \text{ while } B \text{ od }, m ) \rightarrow (C; \text{ while } B \text{ do } C \text{ od }, m )
\]

c. Using the rules given for natural semantics in class, and the rules written in parts a and b, give a proof that starting with a memory that maps x to 3, do y ::= ++x while x < 5 od evaluates to a memory that maps x and y to 5. \\

Natural Semantics Proof: \\

\begin{proof}

\( \text{let } m' = \{ x \rightarrow 4 \}	 \) \\
\( \text{let } m'' = \{ x \rightarrow 4, y \rightarrow 4 \}	 \) \\
\( \text{let } m''' = \{ x \rightarrow 5, y \rightarrow 4 \}	 \) \\
\( \text{let } m'''' = \{ x \rightarrow 5, y \rightarrow 5 \}	 \)

\[
	\frac
	{
		\overbrace{
			(y ::= ++x, m) \Downarrow m'[y \rightarrow 4]
		}^{e_1}
		\hspace*{4mm}
		\overbrace{
			(\while x < 5 \do y ::= ++x  \od, m'') \Downarrow m''''
		}^{e_2}
	}
	{
		(\do y ::= ++x \while x < 5 \od, m) \Downarrow m''''
	}
\]

\[
	\frac
	{	
		\dfrac
		{
			m(x) + 1 = 4
		}
		{
			(++x, m) \Downarrow (4, m[x \rightarrow 4])
		}
	}
	{
		\underbrace{
			(y ::= ++x, m) \Downarrow m'[y \rightarrow 4]
		}_{e_1}
	}
\]

\[
	\dfrac
	{
            	\dfrac
            	{
            		\dfrac
            		{
            			4 < 5 = \text{ true }
            		}
            		{
            			(x, m'') \Downarrow 4 \hspace*{4mm} (5, m'') \Downarrow 5
            		}
            	}
            	{	
        			(x < 5, m'') \Downarrow \text{ true }
            	}
            	\hspace*{4mm}
            	\dfrac
            	{
            		\dfrac
            		{
            			m(x) + 1 = 5
            		}
            		{
            			(++x, m'') \Downarrow (5, m''[x \rightarrow 5 ])
            		}
            	}
            	{	
        			(y ::= ++x, m'') \Downarrow m'''[y \rightarrow 5]
            	}
            	\hspace*{4mm}
            	\dfrac
            	{
            		\dfrac
            		{
            			\dfrac
            			{
            				5 < 5 = \text{ false}
            			}
            			{
            				(x, m'''') \Downarrow 5 \hspace*{4mm} (5, m'''') \Downarrow 5
            			}
            		}
            		{
            			(x < 5, m'''') \Downarrow \text{ false }
            		}
            	}
            	{	
            		(\while x < 5 \do y ::= ++x \od, m'''') \Downarrow m''''
            	}
	}
	{
		\underbrace{
			(\while x < 5 \do y ::= ++x \od, m'') \Downarrow m''''
		}_{e_2}
	}
\]

\end{proof}

\pagebreak
Transition Semantics Proof: \\

\begin{proof}

\( \text{let } m' = \{ x \rightarrow 4 \}	 \) \\
\( \text{let } m'' = \{ x \rightarrow 4, y \rightarrow 4 \}	 \) \\
\( \text{let } m''' = \{ x \rightarrow 5, y \rightarrow 4 \}	 \) \\
\( \text{let } m'''' = \{ x \rightarrow 5, y \rightarrow 5 \}	 \)

\[
	(\do y::= ++x \while x < 5 \od, m) \rightarrow
\]

\[
	\dfrac
	{
		\dfrac
		{
			4 = m(x) + 1
		}
		{
			(++x, m) \rightarrow (4, m[x \rightarrow 4])
		}
	}
	{
		\rightarrow (y ::= ++x ; \while x < 5 \do y ::= ++x \od, m)
	}
\]

\[
	\frac
	{
		(y ::= 4, m') \rightarrow m'[y \rightarrow 4]
	}
	{
		\rightarrow (y ::= 4; \while x < 5 \do y ::= ++x \od, m')
	}
\]

\[
	\rightarrow (\while x < 5 \do y ::= ++x \od, m'')
\]

\[
	\frac
	{
		(x, m'') \rightarrow (4, m'')
	}
	{
		\rightarrow (\iftext x < 5 \thentext y ::= ++x; \while x < 5 \do y ::= ++x \od \elsetext \skip, m'')
	}
\]

\[
	\frac
	{
		\text{ true } = 4 < 5
	}
	{
		\rightarrow (\iftext 4 < 5 \thentext y ::= ++x; \while x < 5 \do y ::= ++x \od \elsetext \skip, m'')
	}
\]

\[
	\rightarrow (\iftext \text{ true } \thentext y ::= ++x; \while x < 5 \do y ::= ++x \od \elsetext \skip, m'')
\]

\[
	\dfrac
	{
		\dfrac
		{
			5 = m''(x) + 1
		}
		{
			(++x, m'') \rightarrow (5, m''[x \rightarrow 5 ])
		}
	}
	{
		\rightarrow (y::= ++x; \while x < 5 \do y ::= ++x \od, m'')
	}
\]

\[
	\frac
	{
		(y ::= 5, m''') \rightarrow m'''[y \rightarrow 5]
	}
	{
		\rightarrow (y ::= 5; \while x < 5 \do y ::= ++x \od, m''')
	}
\]

\[
	\rightarrow (\while x < 5 \do y ::= ++x \od, m'''')
\]

\def \looptext { \while x < 5 \do y ::= ++x \od }

\[
	\frac
	{
		(x, m'''') \rightarrow (5, m'''')
	}
	{
		\rightarrow (\iftext x < 5 \thentext y ::= ++x; \looptext \elsetext \skip, m'''')
	}
\]

\[
	\frac
	{
		5 < 5 = \text{ false }
	}
	{
		\rightarrow (\iftext 5 < 5 \thentext y ::= ++x; \looptext \elsetext \skip, m'''')
	}
\]

\[
	\rightarrow (\iftext \text{ false } \thentext y ::= ++x; \looptext \elsetext \skip, m'''')
\]

\[
	\rightarrow (\skip, m'''') \rightarrow m''''
\]

\end{proof}

\end{enumerate}
		
\end{document}

