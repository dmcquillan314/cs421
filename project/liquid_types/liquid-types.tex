\documentclass[12pt,letterpaper]{article}

\newenvironment{proof}{\noindent{\bf Proof:}}{\qed\bigskip}

\newtheorem{theorem}{Theorem}
\newtheorem{corollary}{Corollary}
\newtheorem{lemma}{Lemma} 
\newtheorem{claim}{Claim}
\newtheorem{fact}{Fact}
\newtheorem{definition}{Definition}
\newtheorem{assumption}{Assumption}
\newtheorem{observation}{Observation}
\newtheorem{example}{Example}
\newcommand{\qed}{\rule{7pt}{7pt}}

\newcommand{\assignment}[4]{
\thispagestyle{plain} 
\newpage
\setcounter{page}{1}
\noindent
\begin{center}
\framebox{ \vbox{ \hbox to 6.28in
{\bf CS421 \hfill #1}
\vspace{4mm}
\hbox to 6.28in
{\hspace{2.5in}\large\mbox{Problem Set #2}}
\vspace{4mm}
\hbox to 6.28in
{{\it Handed Out: #3 \hfill Due: #4}}
}}
\end{center}
}

\newcommand{\solution}[4]{
\thispagestyle{plain} 
\newpage
\setcounter{page}{1}
\noindent
\begin{center}
\framebox{ \vbox{ \hbox to 6.28in
{\bf CS421 : #3 \hfill #4}
\vspace{4mm}
{#1 \hfill {\it Handed In: #2}}
}}
\end{center}
\markright{#1}
}

\newenvironment{algorithm}
{\begin{center}
\begin{tabular}{|l|}
\hline
\begin{minipage}{1in}
\begin{tabbing}
\quad\=\qquad\=\qquad\=\qquad\=\qquad\=\qquad\=\qquad\=\kill}
{\end{tabbing}
\end{minipage} \\
\hline
\end{tabular}
\end{center}}

\def\Comment#1{\textsf{\textsl{$\langle\!\langle$#1\/$\rangle\!\rangle$}}}


\usepackage{algorithm}
\usepackage{listings}
%\usepackage{algpseudocode}
\usepackage{graphicx,amssymb,amsmath}
\usepackage{epstopdf}
\sloppy

\oddsidemargin 0in
\evensidemargin 0in
\textwidth 6.5in
\topmargin -0.5in
\textheight 9.0in

\begin{document}

\solution{Dan McQuillan}{\today}{Liquid Types article}{Summer 2015}

\pagestyle{myheadings}  % Leave this command alone
	 	
% introduction
The verification of low level programs can prove to be challenging due to the presence of mutable state, pointer arithmetic, and unbounded heap-allocated data structures. Verification of these programs is done through the use of dependent refinement types.  Each program variable and expression is represented as a triple of the expression or variable, a conventional type (such as int or bool), and a refinement predicate.

Current implementations of verification rely on a large amount of annotations that would need to be placed in the source files. The presented software, CSolve, is a proposed way to automate this process which is based on the concept of liquid types. Liquid types combine refinement types with three elements to automate verification:

\begin{enumerate}
	\item First, associating the refinement types with heap locations and tracking the locations referenced by pointers.
	\item Second, adding constructs which allow strong updates to the types of heap locations so that the system can verify the correct initialization of newly-allocated regions of memory.
	\item Third, utilize refinement type inference to automatically verify important safety properties without extenuous amounts of annotations.
\end{enumerate}

% Architecture, use and availability
Type inference is split into four phases:

\begin{enumerate}

\end{enumerate}

\end{document}

